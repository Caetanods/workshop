\documentclass[10pt,a4paper]{article}
\usepackage[utf8]{inputenc}
\usepackage[portuguese]{babel}
\usepackage[T1]{fontenc}
\begin{document}

\begin{center}
{\LARGE \textbf{PROGRAMA DO WORKSHOP (incompleto)}}
\end{center}

Programa sob construção...

\section{Introdução}

\begin{enumerate}
\item O que é um modelo?

	\begin{enumerate}
	\item O que são os parâmetros de um modelo?
	\item O básico sobre estimativa de parâmetros de modelos.
	\item Solução exata por derivação (aka. matemágica). (Mostrar exemplo simples de Máxima Verossimilhança.)
	\item Estimativa usando minimização. (Trabalhar o conceito de busca - “Quente ou frio?”)
	\end{enumerate}
	
\item Função de verossimilhança.
	\begin{enumerate}
	\item O que é uma função de verossimilhança?
	\item Usando a função de verossimilhança para buscar a máxima verossimilhança.
	\item Exemplos de busca de verossimilhança.
	\end{enumerate}
	
\item Superfície de verossimilhança.
	\begin{enumerate}
	\item Exemplos de superfícies em trabalhos publicados.
	\item O conceito de pico.
	\item O conceito de vale.
	\item O conceito de “platô” (Likelihood ridge, flat regions).
	\end{enumerate}	
	
\end{enumerate}

\section{A ideia por trás do MCMC.}

\begin{enumerate}
\item Explorando um terreno desconhecido.
	\begin{enumerate}
	\item Analogia: Robôs em Marte.
	\item Distribuição posterior em função da superfície de verossimilhança.
	\item Analogia: Escolhendo chapéis.
	\item Mais tempo olhando os chapéis que melhor serviram.
	\end{enumerate}
\end{enumerate}

\section{Comportamento da estimativa usando máxima verossimilhança.}

\begin{enumerate}
\item Ponto de partida.
	\begin{enumerate}
	\item Ponto de partida pode terminar em ótimo local.
	\item Greed searches.
	\item Múltiplas estimativas para verificar convergência em máxima verossimilhança.	
	\end{enumerate}
	
\item Não tão simples quanto parece.
	\begin{enumerate}
	\item Algorítimos de busca restrita de MLE. (dificuldade de estimar certos parâmetros).
	\item O problema do ótimo local.
	\end{enumerate}
\end{enumerate}

\section{MCMC passo a passo}

Core da parte 1 do curso. Objetivo aqui é aprender os elementos de um Markov chain Monte Carlo. Vamos trabalhar passo a passo para mostrar como o MCMC funciona e a lógica por trás da busca. Vamos utilizar exemplos e experimentos computacionais para demonstrar os conceitos. Fórmulas e provas matemáticas somente serão utilizadas em casos que a intuição nos falha. Essa parte termina quando conseguirmos escrever nossa própria busca de MCMC em R.

\subsection{Desconfundindo o prior}

Nesta parte vamos usar experimentos para entender como a distribuição a priori do parâmetros (prior) influência nas estimativas. Vamos testar diferentes priors sob situação em que o sinal dos dados é forte e em outras quando temos poucos dados. Discussão geral de como proceder e quais testes podemos fazer.

\subsection{Convergência}

Are we there yet? Definição de convergência. Vamos explorar as ferramentas para verificar se nossa cadeia de MCMC atingiu convergência. Vamos discutir quais aspectos de nossa definição cada uma das ferramentas está testando. Vamos implementar no MCMC uma forma simples de continuar a cadeia de onde ela parou, simulando uma análise que precisa rodar por mais tempo.

\section{Modelos filogenéticos comparativos de evolução de traits}

\end{document}